\section{Algebra}

\subsection{Solving Equations}

In order to solve an equation for an unknown variable, we isolate that variable by ``undoing'' all the operations which have been done to that variable. We want the variable on one side of the equation and numbers on the other side. For example, 
to solve
\begin{equation*}
    \log\left(\frac{3x^2}{4} + 1\right) = 12
\end{equation*}
we want to ``unwrap'' $x$ from the outside in. The order of operations on the left-hand side of the equation is: ``first, $x$ is squared, then multiplied by 3, then divided by 4, then added to 1, and then we take the logarithm of it.'' To isolate $x$, we apply the inverse operations in 
the opposite order. The logarithm is applied last, so we first apply the inverse operation: putting $e$ to the power of both sides:
\begin{equation*}
    \begin{aligned}
        e^{\log\left(\frac{3x^2}{4} + 1\right)} & = e^{12} \\
       \frac{3x^2}{4} + 1 & = e^{12} 
    \end{aligned}
\end{equation*}
The second-to-last operation is to add 1; the inverse operation is to subtract 1:
\begin{equation*}
    \begin{aligned}
       \frac{3x^2}{4} + 1 - 1 & = e^{12} - 1 \\ 
       \frac{3x^2}{4} & = e^{12} - 1 
    \end{aligned}
\end{equation*}
Continuing this way, the next few steps are 
\begin{equation*}
    \begin{aligned}
       \frac{3x^2}{4} \cdot 4 & = (e^{12} - 1) \cdot 4 &  (\text{Inverse of dividing by 4 is multiplying by 4}) \\  
        3x^2 & = 4 (e^{12} - 1) \\ 
        \frac{3x^2}{3} & = \frac{4 (e^{12} - 1)}{3}  &  (\text{Inverse of multiplying by 3 is dividing by 3}) \\
        x^2 & = \frac{4}{3} (e^{12} - 1) \\ 
        \sqrt{x^2} & = \sqrt{\frac{4}{3} (e^{12} - 1)} &  (\text{Inverse of squaring is square root}) \\ 
        x & = \sqrt{\frac{4}{3} (e^{12} - 1)}
    \end{aligned}
\end{equation*}

\subsection{Solving Inequalities}

The same rules apply for solving inequalities, but you must switch the direction of the inequality sign whenever you multiply or divide both sides by a negative number. That is, $\ge$ becomes $\le$ and vice versa. Similarly, $\geq$ 
becomes $\leq$ and vice versa. If you are multiplying or dividing by a variable which could be either positive or negative, you must split the inequality into three cases; in Case 1, assume the variable is positive (don't flip the direction 
of the inequality symbol when you multiply/divide by the variable). In Case 2, assume the variable is negative (flip the direction of the inequality symbol when you multiply/divide by the variable). In Case 3, assume the variable is exactly 0 
and check if the inequality holds true. (If this is confusing, don't worry about it too much; this doesn't come up that often.)

In most cases, we can find ways around dividing by a variable. We should do this, because dividing by a variable which could be 0 can cause problems. For an easy example, if trying to solve
\begin{equation*}
    2 x = 3x
\end{equation*}
you can divide by $x$ to get 
\begin{equation*}
    \begin{aligned}
        \frac{2 x}{x} & = \frac{3x}{x} \\ 
        2 & = 3
    \end{aligned}
\end{equation*}
which is obviously not true. The correct solution is
\begin{equation*}
    \begin{aligned}
        2x  & = 3x \\ 
        2x - 2x & = 3x - 2x \\ 
        0 & = x
    \end{aligned}
\end{equation*}
so $x = 0$. This is the correct answer, because we can substitute it into the original equation and see that it is true:
\begin{equation*}
    \begin{aligned}
        2x  & = 3x \\ 
        2(0)  & = 3(0) \\ 
        0 & = 0
    \end{aligned}
\end{equation*}