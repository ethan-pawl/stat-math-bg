\section{Results From Integral Calculus}

There are two types of integrals: \textit{indefinite integrals} (which do not have bounds) and 
\textit{definite integrals} (which have bounds).

Some useful indefinite integrals:
\begin{equation*}
    \begin{aligned}
        & \int 0\ dx = C & \qquad & \int \ dx = x + C & \qquad & \int x^n\ dx = \frac{x^{n+1}}{n + 1} + C,\text{ for } n \neq -1 \\ 
        & \int e^x\ dx = e^x + C & \qquad &\int \frac{1}{x}\ dx = \log |x| + C & \qquad & \int \log x = x \log x - x + C
    \end{aligned}
\end{equation*}
The function (on the right-hand side) we get after taking the indefinite integral is called the \mbox{\textit{antiderivative}}. 

\subsection{The Fundamental Theorem of Calculus}

This theorem is very important (hence its name). The value of a 
definite integral is
\begin{equation*}
    \int_a^b f(x)\ dx = F(x)\Big|_a^b = F(b) - F(a)
\end{equation*}
where $F$ is the antiderivative of $f$. That is, we first find the antiderivative of $f$, evaluate 
it at $x = b$, evaluate it at $x = a$ and the subtract the latter from the former. For example:
\begin{equation*}
    \int_2^4 x^2\ dx = \frac{x^3}{3}\Big|_2^4 = \frac{4^3}{3} - \frac{2^3}{3} 
    = \frac{4^3 - 2^3}{3} = \frac{64 - 8}{3} = \frac{56}{3}
\end{equation*}

\subsection{Integral Properties}

\begin{equation*}
    \begin{aligned}
        & \int f(x) + g(x)\ dx = \int f(x)\ dx + \int g(x)\ dx  \\ 
        & \int c \cdot f(x)\ dx = c \int f(x)\ dx \\
        & \int_a^a f(x)\ dx = 0 \\ 
        & \int_b^a f(x)\ dx = - \int_a^b f(x)\ dx
    \end{aligned}
\end{equation*}

\subsection{Integration Techniques}

\subsubsection{$u$-substitution}

When faced with an integral which does not appear in the above table, such as
\begin{equation*}
    \int x e^{x^2}\ dx
\end{equation*}
we can reduce the complexity by making a substitution. We find a function of $x$ in the \textit{integrand} (the function we're integrating) 
whose derivative also appears in the integrand. This cancels out complicated terms and allows us to simplify it into an integral we know. 
In the above integral, we let $u = x^2$. Notice that $\frac{du}{dx} = 2x$, which is proportional to the factor of $x$ in the integrand. 
Rearranging the integrand, we have 
\begin{equation*}
    \int x e^{x^2}\ dx = \int e^{x^2} x\ dx
\end{equation*}
Making the $u$-substitution, we have
\begin{equation*}
    \int e^{x^2} x\ dx = \int e^u x\ dx
\end{equation*}
But now we have two variables in the integrand; we need to replace all $x$'s with functions of $u$. We do this by substituting $dx$ in terms of $x$ and $du$:
\begin{equation*}
    \begin{aligned}
        \frac{du}{dx} & = 2x \\ 
        du &= 2x dx \\ 
        dx &= \frac{du}{2x} 
    \end{aligned}
\end{equation*}
Substituting this value of $dx$ into the integral gives
\begin{equation*}
    \begin{aligned}
        \int e^u x\ dx & = \int e^u x\ \frac{du}{2x} \\ 
        & = \int \frac{1}{2} e^u\ du 
    \end{aligned}
\end{equation*}
and this integral is easy to solve using the list of integrals at the beginning of this section:
\begin{equation*}
    \begin{aligned}
         \int \frac{1}{2} e^u\ du  & = \frac{1}{2} \int e^u\ du \\ 
         & = \frac{1}{2} e^u + C
    \end{aligned}
\end{equation*}
Finally, we substitute $u = x^2$:
\begin{equation*}
    \frac{1}{2} e^u + C = \frac{1}{2} e^{x^2} + C
\end{equation*}
For a definite integral, there is no need to substitute $u$ in terms of $x$ at the end; instead, just evaluate the bounds in the $u$ space. 
For example, when solving the definite integral
\begin{equation*}
    \int_1^3 x e^{x^2}\ dx
\end{equation*}
we note that the bounds are values of $x$. I will make this clear by writing
\begin{equation*}
    \int_{x = 1}^{x = 3} x e^{x^2}\ dx
\end{equation*}
We substitute $u = x^2$ and get 
\begin{equation*}
    \int_{x = 1}^{x = 3} \frac{1}{2} e^u\ du
\end{equation*}
Here, the bounds are in terms of $x$, but we are integrating with respect to $u$. To resolve this, we find the value of $u$ at each of the $x$ bounds:
\begin{equation*}
    \begin{aligned}
        x = 1 \text{ and } u = 2 x & \Rightarrow u = 2(1) = 2 \\ 
        x = 3 \text{ and } u = 2 x & \Rightarrow u = 2(3) = 6 \\ 
    \end{aligned}
\end{equation*}
so the integral becomes
\begin{equation*}
    \int_{u = 2}^{u = 6} \frac{1}{2} e^u\ du = \int_2^6 \frac{1}{2} e^u\ du
\end{equation*}
We then proceed with the fundamental theorem:
\begin{equation*}
    \begin{aligned}
        \int_2^6 \frac{1}{2} e^u\ du & = \frac{1}{2} \Big[e^u\Big|_2^6 \\ 
        & \boxed{= \frac{1}{2} (e^6 - e^2)}
    \end{aligned}
\end{equation*}

\subsubsection{Integration by Parts}

\begin{equation} \label{eqn:intParts}
    \int u\ dv = u v - \int v\ du
\end{equation}
where $du = u'(x)\ dx$ and $v(x) = \int dv$. The idea is that we should factor a complex integrand into 
a product of two functions: one which is easy to differentiate, and one which is easy to 
integrate. After we apply Equation (\ref{eqn:intParts}), the integral is easy. For example, to find
\begin{equation*}
    \int x e^x\ dx
\end{equation*}
we let $u(x) = x$ and $dv = e^x\ dx$. Then $du = u'(x)\ dx = 1\ dx = dx$ and $
v = \int e^x\ dx = e^x$ (we can ignore the $+C$ as long as we add one at the very end). So 
\begin{equation*}
    \begin{aligned}
        \int x e^x\ dx = \int u\ dv = u v - \int v\ du = x e^x - \int e^x\ dx & = xe^x - e^x + C  \\ 
        & = \boxed{e^x (x - 1) + C}
    \end{aligned}
\end{equation*}
For a definite integral
\begin{equation*}
        \int_a^b x e^x\ dx 
\end{equation*}
we omit the $+C$ and instead apply the fundamental theorem to both terms immediately after applying integration by parts:
\begin{equation*}
    \begin{aligned}
        \int_a^b x e^x\ dx & = \Big[x e^x - \int e^x\ dx \Big|_a^b \\ 
        & = \Big[x e^x - e^x\Big|_a^b \\ 
        & = \Big[e^x(x-1)\Big|_a^b
    \end{aligned}
\end{equation*}
A mnemonic device to remember $u v - \int v\ du$ is ``when you're outside getting $uv$ rays, 
you're not inside playing $v\ du$ games.'' This means, ``when you're outside (of the integral sign) 
getting $uv$ rays, you're not ($-$) inside (the integral sign) playing $v\ du$ games.'' 
(The credit for this saying goes to my high school calculus teacher, left anonymous for privacy.)

\subsection{Integrals of Functions of More Than One Variable}

When integrating a function of more than one variable, we integrate with respect to a single 
variable at a time, holding the others constant and apply the fundamental theorem to the variable 
with respect to which we're integrating (if the integral is definite). For example, 
\begin{equation*}
    \begin{aligned}
        \int_0^1 2x y\ dx & = \int_0^1 \boxed{2x}\ y\ dx \\ 
        & = \Big[x^2 y \Big|_{x = 0}^{x = 1}  \\ 
        & = 1^2 y - 0^2 y \\ 
        & = y
    \end{aligned}
\end{equation*}
For another example, consider the same function integrated with respect to $y$:
\begin{equation*}
    \begin{aligned}
        \int_2^4 2xy\ dx & = \Big[2x \frac{y^2}{2} \Big|_{y = 2}^{y = 4}  \\ 
         & = \Big[x y^2 \Big|_{y = 2}^{y = 4}  \\ 
        & = 4^2 x -  2^2 x \\ 
        & = 16 x - 4 x \\ 
        & = 12x
    \end{aligned}
\end{equation*}

\begin{samepage}
\subsubsection{Iterated Integrals}

    We can also integrate with respect to multiple variables, one after the other. Consider the 
    double integral
    \begin{equation*}
    \int_0^2 \int_1^3 x^2 y^3 + y\ dx\ dy
    \end{equation*}
    We solve this by first considering $y$ as a fixed constant, and using the Fundamental Theorem of 
    Calculus to integrate with respect to $x$:
    \begin{equation*}
        \begin{aligned}
            \int_0^2 \int_1^3 x^2 y^3 + y\ dx\ dy & = \int_0^2 \Big[ \frac{x^3}{3} y^3 + xy \Big|_{x = 1}^3  \ dy \\ 
            & = \int_0^2 \frac{3^3}{3} y^3 + 3y - \left(\frac{1^3}{3} y^3 + 1y\right)\ dy \\ 
            & = \int_0^2 \frac{3^3 - 1^3}{3} y^3 + 3y - y\ dy \\ 
            & = \int_0^2 \frac{27 - 1}{3} y^3 + 2y\ dy \\ 
            & = \int_0^2 \frac{26}{3} y^3 + 2y\ dy \\ 
        \end{aligned}
    \end{equation*}
    and then integrate with respect to $y$ using the fundamental theorem:
    \begin{equation*}
        \begin{aligned}
            \int_0^2 \frac{26}{3} y^3 + 2y\ dy & = \Big[\frac{26}{3} \cdot \frac{y^4}{4} + y^2 \Big|_0^2 \\ 
            & = \frac{26}{3}\cdot  \frac{2^4}{4} + 2^2 - \left(\frac{26}{3} \cdot \frac{0^4}{4} + 0^2\right) \\ 
            & = \frac{13}{3} \cdot \frac{16}{2} + 4 - 0 \\ 
            & = \frac{13}{3} \cdot 8 + 4 \\ 
            & = \boxed{\frac{104}{3} + 4}
        \end{aligned}
    \end{equation*}
\end{samepage}

We can also integrate with respect to $y$ first and then $x$:
\begin{equation*}
    \int_0^2 \int_1^3 x^2 y^3 + y\ dx\ dy = \int_1^3 \int_0^2  x^2 y^3 + y\ dy\ dx
\end{equation*}
This is called \textbf{Fubini's Theorem}. Sometimes, however, the bounds of one variable will 
depend on another variable:
\begin{equation*}
    \int_0^2 \int_0^y x^2 y^3 + y\ dx\ dy
\end{equation*}
Here, we can still change the order of integration, but we will have to adjust the bounds. The 
bounds of the above integral correspond to the system of inequalities
\begin{equation*}
    \begin{cases}
        \begin{aligned}
            0 \leq y \leq 2 \\
            0 \leq x \leq y 
        \end{aligned}
    \end{cases}
\end{equation*}
If we want to integrate with respect to $y$ first, we need to solve this inequality such that $x$ 
is bounded between two constants and $y$ is bounded between a constant and $x$. 
The best way to solve inequalities like this one is by drawing a graph. 
I will not discuss this in further detail here because it can get quite complicated, 
but the solution in this case is
\begin{equation*}
    \begin{cases}
        \begin{aligned}
            0 \leq x \leq 2 \\
            x \leq y \leq 2 
        \end{aligned}
    \end{cases}
\end{equation*}
so 
\begin{equation*}
    \int_0^2 \int_0^y x^2 y^3 + y\ dx\ dy = \int_0^2 \int_x^2 x^2 y^3 + y\ dy\ dx
\end{equation*}