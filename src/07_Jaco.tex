\section{Jacobian Matrices and Determinants}

Consider the system of equations 
\begin{equation*}
    \begin{cases}
        \begin{aligned}
            u &= f_1(x, y) \\ 
            v &= f_2(x, y)
        \end{aligned}
    \end{cases}
\end{equation*}
and imagine we want to integrate 
\begin{equation*}
    \iint h(x, y)\ dx\ dy
\end{equation*}
Sometimes, this integral is much easier to evaluate by first solving for $x$ and $y$ in terms of 
$u$ and $v$ and then integrating with respect to $u$ and $v$. To do this, we first need to solve 
for $x$ and $y$ in terms of $u$ and $v$ to get the system 
\begin{equation*}
    \begin{cases}
        \begin{aligned}
            x &= g_1(u, v) \\ 
            y &= g_2(u, v)
        \end{aligned}
    \end{cases}
\end{equation*}
and then find the Jacobian matrix of partial derivatives, which is
{
    \renewcommand{\arraystretch}{1.5}
    \setlength{\arraycolsep}{6pt}

    \begin{equation*}
        J = 
        \begin{pmatrix}
            \frac{\partial x}{\partial u} & \frac{\partial x}{\partial v} \\ 
            \frac{\partial y}{\partial u} & \frac{\partial y}{\partial v}
        \end{pmatrix}
    \end{equation*}
}
Once we do that, we can evaluate the integral using the change of variables formula:
\begin{equation} \label{eqn:CoV}
    \iint h(g_1(u, v), g_2(u, v))\ \text{abs det}(J)\ du\ dv
\end{equation}
where $\text{abs det}(J)$ means ``the absolute value of the determinant of matrix $J$.'' (I 
will discuss how to calculate determinants later.) 

I will not instruct readers on the solution of this integral (please refer to a calculus textbook), 
but I mention this formula here for two reasons. First, the change of variables is useful in 
probability and makes the necessity of the Jacobian matrix clear. 
Second, a remark about notation. Often, authors write $|J|$ instead of $\text{abs det}(J)$, 
but since $|\cdot|$ is often used to 
denote either the determinant \textit{or} the absolute value, students may think that 
$|J|$ is simply the determinant of the matrix $J$ and forget to take the absolute value. 
I urge students 
to instead write $\text{abs det}$.

\textbf{Example}: Consider the transformation
 \begin{equation*}
    \begin{aligned}
        u &= x + y \\ 
        v &= x - y
    \end{aligned}
\end{equation*}
We can solve these equations by elimination. Adding the two equations gives
 \begin{equation*}
    \begin{aligned}
        u + v &= 2x + y - y = 2x \\ 
        x & = \frac{u + v}{2} = \frac{u}{2} + \frac{v}{2}
    \end{aligned}
\end{equation*}
Subtracting the second equation from the first gives
 \begin{equation*}
    \begin{aligned}
        u - v &= x -  + 2y = 2y \\ 
        y & = \frac{u - v}{2} = \frac{u}{2} - \frac{v}{2}
    \end{aligned}
\end{equation*}
so the inverse transformation is 
 \begin{equation*}
    \begin{cases}
        \begin{aligned}
            x & = \frac{u}{2} + \frac{v}{2} \\ 
            y & = \frac{u}{2} - \frac{v}{2}
        \end{aligned}
    \end{cases}
\end{equation*}
Therefore, the Jacobian matrix is 
{
    \renewcommand{\arraystretch}{1.5}
    \setlength{\arraycolsep}{6pt}
    \begin{equation*}
        J = 
        \begin{pmatrix}
            \frac{\partial x}{\partial u} & \frac{\partial x}{\partial v} \\ 
            \frac{\partial y}{\partial u} & \frac{\partial y}{\partial v}
        \end{pmatrix} = 
        \begin{pmatrix}
            \frac{1}{2} & \frac{1}{2} \\ 
            \frac{1}{2} & - \frac{1}{2}
        \end{pmatrix}
    \end{equation*}
}
    
\subsection{Matrix Determinant}

The determinant of a 2x2 matrix
\begin{equation*}
    A = 
    \begin{pmatrix}
        a & b \\ 
        c & d
    \end{pmatrix}
\end{equation*}
is denoted by $\text{det}(A)$ or $|A|$ and is equal to 
\begin{equation} \label{eqn:det}
        \text{det}(A) = ad - bc
\end{equation}
For formulas regarding determinants of larger matrices, please refer to a linear algebra 
textbook.

Back to the example: applying Equation (\ref{eqn:det}), the determinant of the Jacobian matrix 
is 
\begin{equation*}
    \begin{aligned}
        \text{det}(J) & = \frac{1}{2} (-\frac{1}{2}) - \frac{1}{2} \cdot \frac{1}{2} \\ 
        & = -\frac{1}{4} - \frac{1}{4} \\ 
        & = -\frac{2}{4} \\ 
        & = -\frac{1}{2}
    \end{aligned}
\end{equation*}
Thus, 
\begin{equation*}
    \text{abs det}(J) = \left|- \frac{1}{2}\right| = \frac{1}{2}
\end{equation*}